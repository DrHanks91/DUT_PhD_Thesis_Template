

% set tex block for references
\newcommand*{\englishTOCTitle}{TABLE OF CONTENTS}

\newcommand*{\titleCh}{中文题目}
\newcommand*{\titleEn}{English  Title}



\newcommand*{\titleOdd}{大连理工大学博士学位论文}

% \newcommand*{\authorName}{$\blacksquare\blacksquare\blacksquare$}
\newcommand*{\authorName}{张三}

%\newcommand*{\authorNameEng}{$\blacksquare\blacksquare\blacksquare$}
\newcommand*{\authorNameEng}{San Zhang}

\newcommand*{\cardNumber}{119120110}

%\newcommand*{\tutorName}{$\blacksquare\blacksquare\blacksquare$}
\newcommand*{\tutorName}{李四}

%\newcommand*{\tutorNameEng}{$\blacksquare\blacksquare\blacksquare$}
\newcommand*{\tutorNameEng}{Si Li}
\newcommand*{\majorName}{机械制造及其自动化}
\newcommand*{\approveYear}{2025}	% input year like 2025
\newcommand*{\approveMonth}{09}	% input month like 06
\newcommand*{\approveDay}{01}	% input day like 25
\newcommand*{\approveDate}{\approveYear 年 \approveMonth 月 \approveDay 日}
\newcommand*{\approveLocation}{大连理工大学机械工程学院知方楼-***}


% Load required packages and define document settings
\RequirePackage[left=2.5cm, right=2.5cm, top=3.5cm, bottom=2.5cm,
headheight = 15 pt,
headsep=15pt,
footskip=15pt]{geometry}



\RequirePackage{graphicx}  


% to make figure inside the section
\RequirePackage{placeins}   
\RequirePackage{etoolbox}  % hooking program
\BeforeBeginEnvironment{section}{\FloatBarrier}  % 在 section 之前插入
\AfterEndEnvironment{section}{\FloatBarrier}  %  在 section 之后插入
\BeforeBeginEnvironment{subsection}{\FloatBarrier}  % 在 subsection 之前插入
\AfterEndEnvironment{subsection}{\FloatBarrier}  %  在 subsection 之后插入



\RequirePackage{booktabs} 
\RequirePackage{float}  
\RequirePackage{longtable} 
\RequirePackage{multirow} 
\RequirePackage{makecell} 
\RequirePackage{siunitx}  
\RequirePackage{array}
\newcolumntype{P}[1]{>{\centering\arraybackslash}p{#1}}  % Define P as a centered column



\RequirePackage[normalem]{ulem} 
\RequirePackage{ragged2e}
\RequirePackage{caption}
\RequirePackage{blindtext}
\RequirePackage{amsmath,amssymb,amsfonts}


%% set cite style  and bibliography
\RequirePackage[backend=biber, doi =false, gbpunctin=false, style=gb7714-2015]{biblatex}	% no doi, gbpunctin=faluse -- no // after [C]
\addbibresource{bib/refs.bib}



\DefineBibliographyStrings{english}{in={}}	% -- no in after [C]
\DefineBibliographyStrings{english}{incn={}}	% -- no // after [C]


\AtBeginBibliography{%	%use etoolbox package to make change
	\zihao{5}  % 参考文献列表:五号字
	\setstretch{1.25}      % 1.25倍行距,  use setspace package
	\setlength{\itemsep}{0pt}  % 参考文献列表: 条目之间的距离为0
	\setlength{\parskip}{0pt}  %  参考文献列表:段间距为0
}


\RequirePackage{tabularx} 

% Define autoref names
\AtBeginDocument{%
	\def\figureautorefname{图}
	\def\tableautorefname{表}
	\def\partautorefname{卷}
	\def\appendixautorefname{附录}
	\def\equationautorefname{式}
	\def\Itemautorefname{列表}
	\def\chapterautorefname{章}
	\def\sectionautorefname{节}
	\def\subsectionautorefname{小节}
	\def\subsubsectionautorefname{条目}
	\def\paragraphautorefname{自然段}
	\def\Hfootnoteautorefname{脚注}
	\def\AMSautorefname{式}
	\def\theoremautorefname{理论}
	\def\pageautorefname{页}
}

% Footnotes
\RequirePackage[perpage]{footmisc}	% start numer of footnote with one, not continous with the former page

\renewcommand{\thefootnote}{\textcircled{\arabic{footnote}}}

% Bilingual Captions
\RequirePackage{bicaption}
\captionsetup[figure][bi-first]{name=图}
\captionsetup[table][bi-first]{name=表}
\captionsetup[figure][bi-second]{name=Fig.}	% name of abbr of caption of figure
\captionsetup[table][bi-second]{name=Tab.}	% name of abbr of caption of table
\captionsetup{justification=centering}
\captionsetup[bi-second]{listtype+=Eng}
\captionsetup{labelsep=space} % 将冒号改为空格(效果:图1 标题) 或 labelsep=none % 完全删除分隔符(效果:图1标题)
\captionsetup[figure]{skip=5pt}   % 控制图的标题与图的间距
\captionsetup[table]{skip=5pt}    % 控制表的标题与表的间距



% set distance , not for exact package
\setlength{\textfloatsep}{10pt}  % distance between figure/table and context
\setlength{\floatsep}{5pt}      % distance between same floatings
\setlength{\intextsep}{10pt}      % distance between embedded figure/table and context


% Floating Environments for List of Figures and Tables
\RequirePackage{newfloat}
\DeclareFloatingEnvironment[fileext=lof2]{figureEng}[Figure][List of Figures]
\DeclareFloatingEnvironment[fileext=lof2]{tableEng}[Table][List of Tables]


% Font settings
\RequirePackage{fontspec}
\setmainfont{Times New Roman}
\setCJKmainfont{SimSun}	% 指定宋体

\setCJKfamilyfont{hwxk}{华文行楷}
\newcommand{\huawenxingkai}{\CJKfamily{hwxk}}

\setCJKfamilyfont{hwxh}{华文细黑}
\newcommand{\huawenxihei}{\CJKfamily{hwxh}}



\RequirePackage{unicode-math}
\setmathfont{TeX Gyre Termes Math}



% Plain Page Style  and Page Style 
\RequirePackage{fancyhdr}
\fancypagestyle{plain}{%
	\fancyhf{}
	\fancyhead[CO]{\zihao{5} \titleOdd}	% 显示学校学位论文+ 五号字
	\fancyhead[CE]{\zihao{5} \titleCh}	% 显示论文题目+ 五号字
	\fancyfoot[CO,CE]{\zihao{-5} {- {\thepage} -}}	% use {} to make page number centered and 小五
	\renewcommand{\headrulewidth}{1pt}
}

% TOC
\RequirePackage[titles]{tocloft}
\renewcommand{\cftchapfont}{\normalfont}
\renewcommand{\cftchappagefont}{\normalfont}
\renewcommand{\cftsecfont}{\normalfont}
\renewcommand{\cftbeforechapskip}{0pt}
\renewcommand{\cftbeforesecskip}{0pt}
\renewcommand{\cftbeforesubsecskip}{0pt}
\renewcommand{\cftchapleader}{\cftdotfill{\cftdotsep}}
\renewcommand{\cftsecleader}{\cftdotfill{\cftdotsep}}
\renewcommand\cftdotsep{1.5}


\renewcommand{\cftfigpresnum}{图~}  % “图” + 空格
\addtolength{\cftfignumwidth}{1em} % 预留“图”字宽度

% 在表目录条目前添加“表 ”
\renewcommand{\cfttabpresnum}{表~}  % “表” + 空格
\addtolength{\cfttabnumwidth}{1em} % 预留“表”字宽度

% \begin{wrapfigure} for using in resume with picture
\RequirePackage{wrapfig} 

% set  TABLE OF CONTENTS
\makeatletter
% 定义英文目录标题和输出命令
\newcommand{\encontentsname}{\englishTOCTitle}
\newcommand\entableofcontents{%
	\if@twocolumn\@restonecoltrue\onecolumn\else\@restonecolfalse\fi
	\chapter*{\encontentsname
		\@mkboth{\MakeUppercase\encontentsname}{\MakeUppercase\encontentsname}}%
	\@starttoc{entoc}% 从 .entoc 文件读取
	\if@restonecol\twocolumn\fi
}
% 定义向英文目录添加条目的命令
\newcommand{\addentoc}[2]{%
	\addcontentsline{entoc}{#1}{\protect\numberline{\csname the#1\endcsname}#2}%
}
\makeatother



%% 
\RequirePackage{setspace}	% to set linespread of specifit paragraph 



% Define CountersF
\RequirePackage{enumitem}
\setlist[enumerate]{
	topsep=0pt,      % 列表与上下文的间距
	partopsep=0pt,   % 列表在段落顶部时的额外间距
	itemsep=0pt,     % 项间间距
	parsep=0pt,      % 段落内间距
	leftmargin = 0pt,
	itemindent = 4em,
}


% Chapter Style and do not add blank row in the setting
\ctexset{
	chapter/name={,},	%  删除"第X章"中的"第"和"章"
	chapter/number=\arabic{chapter},
	chapter/beforeskip=0pt,
	chapter/afterskip=10pt,
	chapter/fixskip=true,	
	section/beforeskip=10pt,
	section/afterskip=10pt,
	section/fixskip=true,		
	subsection/beforeskip=10pt,
	subsection/afterskip=10pt,
	subsection/fixskip=true,	
	chapter/format+=\raggedright\zihao{-3}\heiti,	%小三,黑体,左对齐
	section/format+=\raggedright\zihao{4},	%四号,黑体,左对齐
	subsection/format+=\raggedright\zihao{-4},	%小四,黑体,左对齐
	listfigurename = \centerline{图目录},
	listtablename = \centerline{表目录},
	contentsname = \centerline{目录},
}



\RequirePackage{xstring}  % XeLaTeX 支持的字符串处理宏包

% 定义命令:将输入文本的每个字符(除空格外)替换为*
\newcommand{\starify}[1]{%
	\def\inputtext{#1}% 保存原始文本
	% 替换所有非空格字符为*(利用正则表达式匹配)
	\StrRegExpReplace{\inputtext}{[^\s]}{*}[\outputtext]%
	\outputtext% 输出结果
}


\RequirePackage{hyperref}   
\hypersetup{linktoc=all, colorlinks=true, linkcolor=black, urlcolor=black, citecolor=black, bookmarksnumbered, bookmarksopen=true}	% or

%\hypersetup{linktoc=all, colorlinks=true, linkcolor=blue, urlcolor=blue, citecolor=blue, bookmarksnumbered, bookmarksopen=true}	% or
